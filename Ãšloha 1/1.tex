\documentclass[a4paper, 12pt]{article}

\usepackage{amsmath}
\usepackage[total={17cm,25cm}, top=2.5cm, left=2cm, includefoot]{geometry}
\usepackage{enumitem}
\usepackage[T1]{fontenc}

\setlist[itemize]{topsep=0pt}
\setlength{\itemindent}{0cm}
\setlength{\parskip}{9pt}
\setlength\parindent{0pt}

\begin{document}
  \section*{Úloha č. 1: Skladiště}

  \subsection*{Úkol 1.1}
  Řešením tohoto problému jsou chodby mezi každými dvěma po sobě jdoucími souřadnicemi z následujícího výčtu:

  \begin{center}
    $B3, B2, C2, C3, D3, E3, E4, D4, D5, E5$
  \end{center}

  \subsection*{Úkol 1.2}
  Řešením je z bodu z konce procházet skladiště tak, že nestoupíme mimo vyznačené body a při vstupu na každé pole s hodnotou senzoru $s$ zkontrolujeme, zda $|(t_d - p_{len}) - s| \le 5$, kde $p_{len}$ je délka námi uražené cesty a $t_d$ čas, kdy byl robot chycen.

  Pokud dorazíme na začátek, všechny senzory jsme cestou prošli a délka námi uražené cesty je rovna času chycení, pak cestu vypíšeme.

  Řešení je implementováno v souboru \textit{1.1.py}.

  \subsection*{Úkol 1.3}
  Řešením tohoto problému jsou chodby mezi každými dvěma po sobě jdoucími souřadnicemi z následujících výčtů:

  \begin{center}
    $B3, B2, C2, D2, E2, E3, D3, D4, E4$

    $B3, C3, C2, D2, E2, E3, D3, D4, E4$
  \end{center}

  \subsection*{Úkol 1.4}
  Nejprve získáme a (vzestupně podle naměřeného času) setřídíme lokace všech senzorů.

  Bludiště poté budeme od začátku procházet a hledat všechny platné\footnote{V tom smyslu, že nesíme přecházet přes senzory nebo jiné již nalezené cesty.} cesty mezi senzorem 1 a senzorem 2. Pokud najdeme platnou cestu takovou, že její délka odpovídá hodnotě senzoru na jejím konci, do bludiště zapíšeme hodnoty námi uražené cesty a pokusíme se najít cestu mezi senzory 2 a 3. Tento proces opakujeme pro všechny další dvojice vzestupně setřízených lokací senzorů.

  Řešení je implementováno v souboru \textit{1.2.py}.

  \subsection*{Úkol 1.5}
  Pro tento úkol lze vyzkoušet dvojice senzorů podobně jako v řešení pro úkol 1.4 s tím rozdílem, že musíme zkoušet po sobě jdoucí dvojice ve všech různých platných setřízeních.

  Platná setřízení poznáme tak, že se od sebe hodnoty senzorů bezprostředně u sebe neliší o více než 5.
\end{document}
